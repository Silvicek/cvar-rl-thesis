\usepackage{algorithm}
\usepackage{algorithmic}
\usepackage{amsthm}
\usepackage{datetime}
%\usepackage{bbm}


\newdateformat{monthyeardate}{\monthname[\THEMONTH], \THEYEAR}

\newcommand{\todo}[1]{\begingroup
            \color{Red} \textbf{todo:} {#1}
        	\endgroup }

        	
\newcommand{\unclear}[1]{\begingroup
            \color{Orange} \textbf{unclear:} {#1}
        	\endgroup }


% ========== risk ===========
\newcommand{\cvar}{\text{CVaR}}
\newcommand{\acvara}{\alpha\text{CVaR}_\alpha}
\newcommand{\ycvary}{y\text{CVaR}_y}
\newcommand{\var}{\text{VaR}}
\newcommand{\envelope}{\mathcal{U}_{\cvar}(\alpha, P(\cdot | x, a))}


% ========== sets ===========
\newcommand{\cA}{\mathcal{A}}
\newcommand{\cX}{\mathcal{X}}
\newcommand{\cT}{\mathcal{T}}
\newcommand{\cY}{\mathcal{Y}}


% ========== math ===========
\newcommand{\indicator}{\mathbb{1}}
\newcommand{\expect}{\mathop{\mathbb{E}}}
\newcommand{\expval}[1]{\mathbb{E}\left[ {#1} \right]}
\newcommand{\dt}{\text{d}}
\newcommand{\real}{\mathbb{R}}
\newcommand{\cca}{\dot{=}}
\newcommand{\argmax}{\text{arg}\max}
\newcommand{\argmin}{\text{arg}\min}


% ========== theorems ===========
\newtheorem{theorem}{Theorem}
\newtheorem{corollary}{Corollary}
\newtheorem{lemma}{Lemma}
\newtheorem{definition}{Definition}
\newtheorem{proposition}{Proposition}

% ========== refs ===========
\newcommand{\eqnref}[1]{(\ref{eqn:#1})}
\newcommand{\secref}[1]{Section \ref{sec:#1}}
\newcommand{\figref}[1]{Figure \ref{fig:#1}}
\newcommand{\chref}[1]{Chapter \ref{ch:#1}}
\newcommand{\algref}[1]{Algorithm \ref{alg:#1}}
\newcommand{\thmref}[1]{Theorem \ref{thm:#1}}



% ========== brackets ===========
\newcommand{\bround}[1]{\left( {#1} \right)}
\newcommand{\bsquare}[1]{\left[ {#1} \right]}
\newcommand{\braces}[1]{\left\{ {#1} \right\}}

% ========== algorithm ===========
\newlength\myindent
\setlength\myindent{1em}
\newcommand\bindent{%
  \begingroup
  \setlength{\itemindent}{\myindent}
  \addtolength{\algorithmicindent}{\myindent}
}
\newcommand\eindent{\endgroup}


\newcommand{\given}[1][]{\:#1\vert\:}
\newcommand{\interpI}{\mathcal{I}}







