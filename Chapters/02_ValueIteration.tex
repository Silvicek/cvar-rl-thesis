%************************************************
\chapter{Value Iteration with CVaR}\label{ch:introduction}
%************************************************

\citet{chow2015risk} have recently proposed an approximate value iteration algorithm for CVaR MDPs. This algorithm requires the computation of a linear program in each step of the value iteration procedure. We utilize a connection between the used $\alpha \cvar_\alpha$ function and the quantile function and propose a linear-time algorithm that substitutes the LP computation, making the CVaR value iteration feasible for much larger MDPs. ***experiments*** The procedure also opens the door for a Q-learning variant of the algorithm.


%*****************************************

\section{CVaR Bellman equation}

%*****************************************

\section{Efficient computation}


\subsection{CVaR Value Iteration}

The results of \cite{chow2015risk} heavily rely on the CVaR decomposition theorem, which we show below

\begin{equation}\label{eq:cvardecomp}
\begin{split}
CVaR_\alpha(x, a)&=\min_{\xi} \sum_{x'} p(x, a, x')\xi(x') CVaR_{\xi\alpha}(x')\\
\text{s.t.}\quad &\sum_{x'}p(x, a, x')\xi(x')=1\\
&0 \le \xi(x')\le \dfrac{1}{\alpha}
\end{split}
\end{equation}

The theorem states that we can compute the $CVaR_\alpha(x)$ as the minimal weighted combination of $CVaR_\alpha(x')$ under a perturbed distribution. \cite{chow2015risk} extended this result to a dynamic programming formulation

\begin{equation}
CVaR_\alpha(x)=\max_a R(x, a) + \gamma CVaR_{\alpha}(x, a)
\end{equation}

and showed that the VI procedure is a contraction and perserves the convexity of $\alpha CVaR_\alpha$. The fixed point of this contraction is then the exact CVaR value.

This algorithm is unfortunately unusable in practice, as the state-space is continuous in $\alpha$. The solution proposed in \cite{chow2015risk} is then to represent the convex $\alpha CVaR_\alpha$ as a piecwise linear function. 

***I definition***

The interpolated Bellman operator is then also a contraction and has a bounded error

\begin{equation}
\begin{split}
CVaR_\alpha(x, a)&=\min_{\xi} \sum_{x'} p(x, a, x')\dfrac{I_{x'}(\alpha \xi(x'))}{\alpha}\\
\text{s.t.}\quad &\sum_{x'}p(x, a, x')\xi(x')=1\\
&0 \le \xi(x')\le \dfrac{1}{\alpha}
\end{split}
\end{equation}

\subsection{Quantile representation}

We use the following two facts: firstly, any disrete distribution function has a piecewise linear $\alpha CVaR_\alpha$ function \cite{rockafellar2000optimization}; secondly the $\alpha CVaR_\alpha$ and the quantile function can be computed from each other by utilizing the relation

\begin{equation}
\dfrac{\partial}{\partial \alpha} \alpha CVaR_\alpha(Z) = \dfrac{\partial}{\partial \alpha} \int_0^\alpha VaR_\beta(Z) d\beta = VaR_\alpha(Z)
\end{equation}

***integration constant***

We propose the following improvement: instead of using linear programming for the CVaR computation, we instead use the distributions represented by the $\alpha CVaR_\alpha$ function. 

The computation of CVaR of a discrete probability mixture is a linear-time process as we show bellow. The general steps of the computation are as follows

\begin{enumerate}
\item transform $\alpha CVaR_\alpha$ of each possible state transition to a discrete probability distribution function
\item combine these to to a distribution representing the full state-action distribution
\item compute $\alpha CVaR_\alpha$ for all atoms
\end{enumerate}




%\begin{algorithm}
%\caption{CVaR computation}
%
%\begin{multicols}{2}
%
%\begin{algorithmic}
%    \STATE \textbf{input} $\alpha, x_t, \pi_\text{old}, \gamma$
%    \WHILE{$x_t$ is not terminal}
%    	\STATE $a = \text{arg}\max_a CVaR_\alpha(Z(x_t, a))$
%		\STATE $s = VaR_\alpha(Z(x_t, a))$
%		\columnbreak
%
%    	\STATE $x_t, r_t = \text{envTransition}(x_t, a)$
%    	\STATE $\alpha = F_{Z(x_t, \pi_\text{old}(x_t))}\left(\dfrac{s - r}{\gamma}\right) $ \textcolor{gray}{\# $VaR_\alpha(Z(x_t, \pi_\text{old}(x_t))) == \dfrac{s - r}{\gamma}$}
%   	\ENDWHILE
%\end{algorithmic}
%\end{multicols}
%\end{algorithm}
%
%\begin{minipage}[t]{0.5\linewidth}
%  \vspace{0pt}  
%  \begin{algorithm}[H]
%    \caption{Algo 1}
%    line 1\;
%    line 2\;
%  \end{algorithm}
%\end{minipage}%
%\begin{minipage}[t]{5cm}
%  \vspace{0pt}
%  \begin{algorithm}[H]
%    \caption{Algo 1}
%    line 1\;
%  \end{algorithm}
%\end{minipage}


\subsection{Proof}

%*****************************************

\section{Experiments}



%*****************************************
%*****************************************
%*****************************************
%*****************************************
%*****************************************
